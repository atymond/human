\documentclass{article}

% BASE
\usepackage[hidelinks]{hyperref}

\usepackage{polski}
\usepackage[utf8]{inputenc}
\usepackage{indentfirst}

\usepackage{geometry} %tmargin=3cm, bmargin=3cm, 
\newgeometry{lmargin=3cm, rmargin=3cm, tmargin=2.8cm, bmargin=2.8cm}

% INTERLINIA
\linespread{1.3} %wartosć interlinii
\usepackage{setspace}


% POLSKIE NUMEROWANIE SEKCJI
\makeatletter
 \renewcommand\@seccntformat[1]{\csname the#1\endcsname.\quad}
 \renewcommand\numberline[1]{#1.\hskip0.7em}
\makeatother


\setcounter{secnumdepth}{3} % sections are level 1
\setcounter{tocdepth}{3} % table of contents depth


\renewcommand{\figurename}{Rycina}


% UML DIAGRAMS
%\usepackage{tikz}
%\usetikzlibrary{shapes}
%\usetikzlibrary{arrows.meta}
%\usepackage{tikz-uml}


% DEFINE COLORS
\usepackage{color}
\definecolor{ziel}{RGB}{0, 80, 0}

% ITEMKI
\newenvironment{itemki}{
\begin{itemize}
  \setlength{\itemsep}{1pt}
  \setlength{\parskip}{0pt}
  \setlength{\parsep}{0pt}
}{\end{itemize}}


\usepackage{graphicx} % to use "include graphics"
\usepackage{float} % to use "H" in figures
%\usepackage{adjustbox}
\usepackage{array} % to use "m" in tables


\title{\Huge{Protesty na życzenie}\\ %\LARGE{Projekt prowadzenia działalności}
}
\author{\large{Ahmed Abdelkarim, Agata Cacko, Aleksandra Hernik}}
\date{\large{27.05.2017}}


\begin{document}
\maketitle
\vspace{2cm}
\tableofcontents
\newpage




\section{Streszczenie}
\begin{enumerate}
% mogą to byś subsectiony, ale streszczenie powinno chyba być krótkie
\item Charakterystyka firmy
\item Prezentowana usługa % "informacje o produkcie"
\item Rynek działania
\item Sprzedaż i marketing
\item Analiza procesu wytwórczego % wtf? ze jak to będziemy robić?
\item Analiza finansowa
\item Analiza ryzyka
\end{enumerate}

\section{Opis przedsięwzięcia}
%2.	Opis przedsięwzięcia, jego zakres i uwarunkowania prawne, wybór (wraz z uzasadnieniem) formy prawnej, zespół projektowy
% spółka tamta jedna jedyna dobra, z 5 tys. wkładu?

\section{Analiza rynkowa przedsięwzięcia}
\subsection{Cele organizacji}
\subsection{Opis produktu}
% Opis produktu/usługi, jego odrębność, wskazanie źródeł przewagi konkurencyjnej

\subsection{Charakterystyka branży}
% (analiza otoczenia i analiza konkurencyjna)
% w sensie wszystko w otoczeniu
\subsubsection{Otoczenie bliższe} % te inne firmy, które robią VR
\subsubsection{Otoczenie dalsze} % ekonomia i polityka -> że jak spróbujemy to zrobić w Etiopii, to chyba nie wypali
\subsection{Analiza SWOT}
\subsection{Aspekty prawne}
%Prawne aspekty rozwoju pomysłów i produktów w organizacji


\section{Finansowanie}
%Zarządzanie finansami przedsięwzięcia
\subsection{Nakłady inwestycyjne i źródła finansowania}
\subsection{Prognozy finansowe}
%b.	Prognozy finansowe (min 2 wybrane elementy spośród omawianych na zajęciach metod i narzędzi analizy finansowej)


\section{Ochrona prawna}
\subsection{Podstawowe elementy ochrony prawnej pomysłów, produktów i usług}
\subsection{Ochrona własności intelektualnej w organizacji}


\section{Ograniczenia projektu}
%a.	Wskazanie i oszacowanie ryzyka projektu


\section{Załączniki}


\end{document}












