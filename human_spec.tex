\documentclass{article}

% BASE
\usepackage[hidelinks]{hyperref}

\usepackage{polski}
\usepackage[utf8]{inputenc}
\usepackage{indentfirst}

\usepackage{geometry} %tmargin=3cm, bmargin=3cm, 
\newgeometry{lmargin=3cm, rmargin=3cm, tmargin=2.8cm, bmargin=2.8cm}

% INTERLINIA
\linespread{1.3} %wartosć interlinii
\usepackage{setspace}


% POLSKIE NUMEROWANIE SEKCJI
\makeatletter
 \renewcommand\@seccntformat[1]{\csname the#1\endcsname.\quad}
 \renewcommand\numberline[1]{#1.\hskip0.7em}
\makeatother


\setcounter{secnumdepth}{3} % sections are level 1
\setcounter{tocdepth}{3} % table of contents depth


\renewcommand{\figurename}{Rycina}


% UML DIAGRAMS
%\usepackage{tikz}
%\usetikzlibrary{shapes}
%\usetikzlibrary{arrows.meta}
%\usepackage{tikz-uml}


% DEFINE COLORS
\usepackage{color}
\definecolor{ziel}{RGB}{0, 80, 0}

% ITEMKI
\newenvironment{itemki}{
\begin{itemize}
  \setlength{\itemsep}{1pt}
  \setlength{\parskip}{0pt}
  \setlength{\parsep}{0pt}
}{\end{itemize}}


\usepackage{graphicx} % to use "include graphics"
\usepackage{float} % to use "H" in figures
%\usepackage{adjustbox}
\usepackage{array} % to use "m" in tables


\title{\Huge{Manifeo}\\ %\LARGE{Projekt prowadzenia działalności}
}
\author{\large{Ahmed Abdelkarim, Agata Cacko, Aleksandra Hernik}}
\date{\large{27.05.2017}}


\begin{document}
\maketitle
\vspace{2cm}
\tableofcontents
\newpage




\section{Streszczenie}
\begin{enumerate}
% mogą to byś subsectiony, ale streszczenie powinno chyba być krótkie
\item Charakterystyka firmy
\item Prezentowana usługa % "informacje o produkcie"
\item Rynek działania
\item Sprzedaż i marketing
\item Analiza procesu wytwórczego % wtf? ze jak to będziemy robić?
\item Analiza finansowa
\item Analiza ryzyka
\end{enumerate}

\section{Opis przedsięwzięcia}
Proponowana organizacja zajmowałaby się kompleksową organizacją zgromadzeń publicznych. Docelowym obszarem działań byłaby wstępnie Warszawa, a dalej Polska, z możliwością rozwoju do skali światowej.   

Najkorzystniejszą formą prawną wydaje się tak zwana spółka z o.o. komandytowa, czyli założenie dwóch spółek:
\begin{itemize}
\item Spółka komandytowa, której komandytariuszami są wszyscy wspólnicy,
\item Spółka z ograniczoną odpowiedzialnością, będąca komplementariuszem powyższej spółki komandytowej.
\end{itemize}
W tej konstrukcji spółka komandytowa zajmuje się prowadzeniem samej działalności (w tym generowaniem zysków), a spółka z o.o. zajmuje się zarządzaniem i administrowaniem nią. Dzięki takiemu połączeniu można uniknąć podwójnego opodatkowania występującego w typowych spółkach z o.o., które wypłacają dywidendy (podatek 19\% CIT od dochodów spółki oraz kolejny podatek 19\% CIT od dywidend) -- spółka komandytowa nie podlega podatkowi dochodowemu od osób prawnych, ponieważ jest spółką osobową. Ponadto, spółka z o.o. jako komplementariusz sprawia, że rzeczywista odpowiedzialność finansowa jest ograniczona. Wadami tego rozwiązania jest konieczność prowadzenia pełnej księgowości ze względu na spółkę z o.o., oraz konieczność dwukrotnego wniesienia opłaty rejestracyjnej.

%2.	Opis przedsięwzięcia, jego zakres i uwarunkowania prawne, wybór (wraz z uzasadnieniem) formy prawnej, zespół projektowy

\section{Analiza rynkowa przedsięwzięcia}
\subsection{Cele organizacji}
Początkowym celem organizacji jest zyskanie popularności oraz zaufania klientów -- ze względu na poufny charakter działań, zaufanie jest kluczowe dla dalszych sukcesów. Dalszym celem jest rozszerzenie zasięgu działania firmy -- najpierw na całą Polskę, później na Europę, a ostatecznie na cały świat (z wyjątkiem krajów, w których taka działalność jest niebezpieczna lub nielegalna). Dodatkowym celem jest stworzenie systemu informatycznego, który automatyzowałby wiele z zadań firmy.
\subsection{Opis produktu}
Oferowaną przez firmę usługą jest organizowanie zgromadzeń publicznych takich jak manifestacje, obejmujące wybrane przez klienta elementy spośród:
\begin{itemize}
\item zawiadomienia odpowiednich organów o zgromadzeniu,
\item pomocy przy wybraniu formy zgromadzenia,
\item pomocy przy ustaleniu czasu i lokalizacji (lub trasy),
\item pomocy przy napisaniu oświadczenia obrazującego powód, cel, stanowisko oraz żądania,
\item pomocy w ewentualnych negocjacjach,
\item zbierania danych o temacie zgromadzenia,
\item promowania wydarzenia: 
	\begin{itemize}
	\item w mediach społecznościowych,
	\item w mediach lokalnych (prasa, radio, internet),
	\item poprzez plakaty i ulotki.
	\end{itemize}
\item wymyślenia haseł zgodnych z celem klienta,
\item przygotowania materiałów dla uczestników (np. flagi, transparenty i tabliczki z hasłami),
\item zapewnienia sprzętu (megafon, ale też np. rzutnik i nagłośnienie),
\item pomocy w koordynacji zgromadzenia,
\item organizacji transportu i ewentualnych noclegów dla uczestników zgromadzenia,
\item organizacja cateringu dla uczestników,
\item (potencjalnie) wynajęcia aktorów w celu zawyżenia liczby osób oraz dodania entuzjazmu zgromadzonym.
\end{itemize}
Wszelkie zaangażowanie firmy w protest byłoby utrzymane w tajemnicy. Możliwa byłaby zniżka dla klientów, którzy zgadzają się na rozpowszechnienie informacji o udziale firmy w organizacji zgromadzenia. 

Firma będzie miała stronę internetową i reklamowała się głównie za pomocą mediów społecznościowych. W celu realizacji wyżej wymienionych elementów, wraz ze wzrostem skali przedsięwzięcia stworzony zostanie system informatyczny do wewnętrznego, który automatyzuje wysyłanie zawiadomień o zgromadzeniu, a także w dużym stopniu organizację cateringu, transportu i noclegów. Ponadto, system podpowiada najlepsze opcje dotyczące czasu i lokalizacji lub trasy zgromadzenia, oraz wstępne sugestie dotyczące sloganów.

W Polsce nie istnieją żadne firmy zajmujące się takimi usługami (a przynajmniej nie są one wyszukiwalne za pomocą Google), co daje bardzo dużą przewagę. Firmy zajmujące się zbliżoną działalnością można znaleźć w Stanach Zjednoczonych, na przykład \textit{Crowds On Demand}, \textit{Crowds For Rent} -- firmy specjalizujące się w wynajmowaniu aktorów. W internecie pojawiła się ponadto strona \textit{Demand Protest} -- opisuje firmę, która miała specjalizować się w podobnych usługach -- jednak jest to fikcyjna organizacja, a celem strony było sprowokowanie oburzenia w Amerykańskich mediach. 

Ponieważ tego typu organizacji nie ma w Polsce, a na świecie rzeczywiste firmy nie oferują tak kompleksowej organizacji protestów, a jedynie skupiają się na wynajmowaniu aktorów, działalność ma bardzo dobre perspektywy zarówno na poziomie krajowym, jak i, w dalszych etapach rozwoju firmy, światowym. 
% Opis produktu/usługi, jego odrębność, wskazanie źródeł przewagi konkurencyjnej

\subsection{Charakterystyka branży}
% (analiza otoczenia i analiza konkurencyjna)
% w sensie wszystko w otoczeniu
\subsubsection{Otoczenie bliższe} % te inne firmy, które robią VR
\subsubsection{Otoczenie dalsze} % ekonomia i polityka -> że jak spróbujemy to zrobić w Etiopii, to chyba nie wypali
\subsection{Analiza SWOT}
\subsection{Aspekty prawne}
Ustawami, które są powiązane z prowadzoną działalnością, są przede wszystkim prawa dotyczące zgromadzeń publicznych. Artykuł 57. Konstytucji Rzeczypospolitej Polskiej zapewnia wolność zgromadzeń (zarówno w charakterze czynnym, jak i biernym) -- \textit{Każdemu zapewnia się wolność organizowania pokojowych zgromadzeń i uczestniczenia w nich. Ograniczenie tej wolności może określać ustawa.} Ostatnie zdanie artykułu wskazuje, że ustawy spoza konstytucji mogą ograniczać tą wolność. 

Dokładne warunki zgromadzeń publicznych są specyfikowane przez ustawę z dnia 24 lipca 2015 roku -- prawo o zgromadzeniach (Dz.U. 2015 poz. 1485). Najistotniejszą jest konieczność powiadomienia organu gminy (lub organów wszystkich gmin, na których terenie ma odbywać się zgromadzenie) o organizacji zgromadzenia w terminie od 6 do 30 dni przed jego datą. Jeśli zgromadzenie nie będzie powodować utrudnień w ruchu drogowym, można skorzystać z uproszczonego postępowania -- wtedy zawiadomienie może być dostarczone nie później niż 2 dni przed datą zgromadzenia. Ustawa opisuje dokładne informacje, które muszą znajdować się w zawiadomieniu. Istotnym elementem ustawy jest konieczność posiadania przez organizatora elementów wyróżniających go, a także konieczność rozwiązania zgromadzenia, jeśli uczestnicy nie podporządkują się jego poleceniom lub naruszają przepisy tej ustawy albo karne. Ponadto ważna jest nowelizacja prawa o zgromadzeniach, która zakłada między innymi, że odległość między zgromadzeniami nie może być mniejsza niż 100 metrów, a także przewiduje możliwość otrzymania obowiązującej przez trzy lata zgody na cykliczne organizowanie zgromadzeń z wyłącznością na dane miejsce i czas. 
Jedną z oferowanych usług jest zawiadomienie odpowiednich organów o zgromadzeniu -- w przypadku, gdy klient nie zamawia tej usługi, warunkiem koniecznym współpracy z klientem będzie wykonanie tej czynności przez niego.


Istotnym aktem prawnym jest również artykuł 52. kodeksu wykroczeń, zakazujący organizowania niepokojowych zgromadzeń, przeszkadzania w innych zgromadzeniach, umyślnego niepodejmowania środków niezbędnych, żeby zgromadzenie odbywało się zgodnie z prawem i bez szkód powstałych z winy jego członków, a także podżegania oraz pomocnictwa. W związku z tym, przy organizacji i w trakcie przebiegu zgromadzeń, w przypadku jego przekształcenia się w nielegalne zgromadzenie, firma wycofa swoje wsparcie. 

Firma nie będzie się również angażowała w zgromadzenia, które łamią prawa ograniczające wolność wypowiedzi, takie jak:
\begin{itemize}
\item Artykuły 196, 200b, 202, 212, 216, 255, 255a, 256 i 257 kodeksu karnego, 
\item Ustawa z dnia 5 sierpnia 2010 o ochronie informacji niejawnych (Dz.U. 2010 nr 182 poz. 1228),
\item Prawa autorskie.
\end{itemize}
%Prawne aspekty rozwoju pomysłów i produktów w organizacji


\section{Finansowanie}
%Zarządzanie finansami przedsięwzięcia
\subsection{Nakłady inwestycyjne}
Wydatki można podzielić na jednorazowe, okresowe (comiesięczne, coroczne) i dokonywane na potrzeby konkretnych zleceń.

\subsubsection{Wydatki jednorazowe}
Wspólnicy będą korzystać z własnych komputerów, więc na początku nie będzie konieczności ponoszenia dodatkowych wydatków związanych z ich zakupem.
\begin{table}[!ht]
\label{table2}
\vspace{0.3cm}
\hspace{-2cm}
\centering
\begin{tabular}{|c|c|c|}
\hline
Zasób & Liczba & Koszt \\
\hline
Megafon & 10 & 1000 \\
Drukarka & 1 & 5000 \\
Strona internetowa & 1 & 2000 \\
Znak słowny & 1 & 880 \\
\hline
\end{tabular}
\end{table}

\subsubsection{Wydatki okresowe}
Ze względu na to, że początkowo dochody firmy nie będą stałe, zdecydowaliśmy się na razie nie wynajmować biura i pracować w prywatnych mieszkaniach.

\begin{table}[!ht]
\label{table2}
\vspace{0.3cm}
\hspace{-2cm}
\centering
\begin{tabular}{|c|c|}
\hline
Zasób & Średni miesięczny koszt \\
\hline
Księgowość & 600 \\
Serwisowanie sprzętu & 20 \\
Utrzymanie domeny & 8 \\
Znak słowny & 14 \\
\hline
\end{tabular}
\end{table}

\subsubsection{Wydatki zależne od zlecenia}
\begin{table}[!ht]
\label{table2}
\vspace{0.3cm}
\hspace{-2cm}
\centering
\begin{tabular}{|c|c|}
\hline
Wydatek & Koszt \\
\hline
Nocleg & 30-50/os \\
Catering & 15-30/os \\
Transparent & 0.80/metr \\
Ulotka & 0.30/sztuka \\
Transport (około 16 os) & 0.25/osobokilometr \\
Transport (około 50 os) & 0.12/osobokilometr \\
\hline
\end{tabular}
\end{table}

\subsection{Źródła finansowania}
Wspólnicy dysponują w sumie kwotą 30000 zł. Jest to kwota pozwalająca pokrycie początkowych wydatków i funkcjonowanie firmy przez co najmniej 6 miesięcy.

\subsection{Prognozy finansowe}
%b.	Prognozy finansowe (min 2 wybrane elementy spośród omawianych na zajęciach metod i narzędzi analizy finansowej)
Ze względu na dużą różnorodność w zakresie i skali oferowanych usług trudno przedstawić jednoznaczny cennik naszych usług. W związku z tym został wybrany model polegający na oferowaniu klientom spersonalizowanych ofert, dokładnie odpowiadających ich wymaganiom. Wadą tego rozwiązania jest trudność w przewidzeniu zysków. Z drugiej strony ułatwi to opieranie ceny usług firmy na jej stanie finansowym.

\subsubsection{Rachunek zysków i strat}
W każdym miesiącu w samej Warszawie odbywa się kilka do nawet kilkudziesięciu demonstracji, marszów, protestów i manifestacji. Przyjmijmy, że na początku naszej działalności uda nam się pozyskać jednego klienta w każdym miesiącu.

\begin{enumerate}
\item Przychody

\item Koszty operacyjne
\begin{itemize}
\item Wszystkie wydatki okresowe 
\end{itemize}

\end{enumerate}

\section{Ochrona prawna}
\subsection{Podstawowe elementy ochrony prawnej pomysłów, produktów i usług}
Nie mogą być podjęte żadne działania mające na celu ochronę naszego pomysłu. Prawa autorskie dotyczą tylko utworów, a patenty -- wynalazków. 

Działalność firmy nie wymaga wiedzy specjalistycznej, więc ochrona know-how nie zostanie zastosowana.

Jako że w chwili obecnej nie istnieje konkurencja w tej dziedzinie, nie ma również potrzeba zabezpieczania się przed odejściem pracowników do konkurencyjnych firm.

Nazwa firmy będzie chroniona znakiem słownym zarejestrowanym dla klas 35, 39, 43, 45.
\subsection{Ochrona własności intelektualnej w organizacji}
W Europie nie ma możliwości opatentowania oprogramowania. W związku z tym jedynie prawa autorskie wszystkich narzędzi informatycznych powstałych podczas działalności spółki zostaną przeniesione na spółkę.

\section{Ograniczenia projektu}
%a.	Wskazanie i oszacowanie ryzyka projektu


\section{Załączniki}


\end{document}












